\section{Hoja 2: Resolución numérica de sistemas lineales}

Para la resolución de los sistemas lineales dados para los apartados 1 y 2 implementamos los métodos de Jacobi y de Gauss-Seidel como funciones. Ambas reciben como parámetros la matriz A con los coeficientes y la matriz columna B con los valores resultado del sistema de ecuaciones Ax = B, una solución inicial \textbf{``sol''} para la iteración y el valor de tolerancia. Y ambas devuelven un vector solución y el número de iteraciones efecturadas, que nos servirá para medir la eficiencia del método.

\begin{lstlisting}[language=Matlab, caption={Método de Jacobi.},captionpos=b,texcl=true]
function [res,count] = Jacobi(A, B, sol, tolerancia)
    next=zeros(1,length(sol));
    for i = 1:length(A)
        suma = 0;
        for j = 1:length(sol)
            if(i ~= j)
                suma = suma + A(i,j)*sol(j);
            end
        end
        next(i) = (B(i)-suma)/A(i,i);
    end
    count = 1;
    other = 0;
    for i = 1:length(sol)
        if(abs(sol(i)-next(i)) > tolerancia)
            [next, other] = Jacobi(A,B, next,tolerancia);
            break
        end
    end
    count = count + other;
    res = next;
end
\end{lstlisting}

\begin{lstlisting}[language=Matlab, caption={Método de Gauss-Seidel.},captionpos=b,texcl=true]
function [res, count] = GaussSeidel(A, B, sol, tolerancia)
    next = sol;
    for i = 1:length(A)
        suma = 0;
        for j = 1:length(next)
            if(i ~= j)
                if(i > j)
                    suma = suma + A(i,j)*next(j);
                else
                   suma = suma + A(i,j)*sol(j);
                end
            end
        end
        next(i) = (B(i)-suma)/A(i,i);
    end
    count = 1;
    other = 0;
    for i = 1:length(next)
        if(abs(sol(i)-next(i)) > tolerancia)
            [next, other] = GaussSeidel(A,B,next, tolerancia);
            break
        end
    end
    count = count + other;
    res = next;
end
\end{lstlisting}

\subsection{Apartado 1}

Para el sistema de ecuaciones dado (que no incluiremos aquí por brevedad) la solución obtenida con ambos métodos es \[x_1=3 , x_2=-2.5 , x_3=7\] pero con Gauss-Seidel obtenemos la solución en 3 iteraciones mientras que con el método de Jacobi usamos 4 con la misma toleracia de 0.01.

\subsection{Apartado 2}

Para el segundo sistema de cuatro ecuaciones con cuatro incógnitas podemos observar que Gauss-Seidel se aproxima a la solución de manera mucho más rápida y obtiene mejores resultados al reutilizar los valores de las variables ya computadas durante la iteración.

La solución dada en el enunciado es: \[x_1=2.7273 , x_2 = 0.4040 , x_3 = 0.6364 , x_4 = 0.1919\] es la usada para computar el error.
\begin{table}
\begin{center}
\begin{tabular}{ |c|c|c|c| } 
 \hline 
 Metodo & Tolerancia & Iteraciones (k) & max(Error\textsuperscript{k}) \\ 
 \hline \hline
 Jacobi &  0.1 & 5 & > 100 \\
 \hline
 Jacobi &  0.01 & 12 & 17.46\\
 \hline
 Jacobi &  0.001 & 21 & 1.19\\
 \hline
 Jacobi &  0.0001 & 29 & 0.1155\\
 \hline
 Gauss-Seidel & 0.1 & 4 & 31.9828\\
 \hline
 Gauss-Seidel & 0.01 & 8 & 3.0241\\
 \hline
 Gauss-Seidel & 0.001 & 12 & 0.3491\\
 \hline
 Gauss-Seidel & 0.0001 & 16 &  0.0335\\
 \hline
\end{tabular}
\end{center}
\caption{Interaciones y error según la tolerancia para Jacobi y Gauss-Seidel.}
\end{table}