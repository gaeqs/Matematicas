\section{Hoja 1: Integración numérica}

Para el apartado de integración numérica se han implementado los métodos del rectángulo (por la derecha, izquierda y punto medio), del trapecio y de Simpson.

\begin{lstlisting}[language=Matlab, caption={Fórmulas de integración.},captionpos=b,texcl=true]
function res = rectanguloA(f,a,b) % Rectángulo izquierda.
    res = f(a)*(b-a);
end

function res = rectanguloB(f,a,b) % Rectángulo derecha.
    res = f(b)*(b-a);
end

function res = rectanguloAB(f,a,b) % Rectángulo punto medio.
    res = f((a+b)/2)*(b-a);
end

function res = trapecio(f,a,b) % Fórmula del Trapecio
    res = ((b-a)/2)*(f(a)+f(b));
end

function res = simpson(f,a,b) % Fórmula de Simpson 1/3 
    res = (b-a)*(f(a)+4*f((a+b)/2)+f(b))*1/6;
end
\end{lstlisting}

Adicionalmente se ha implementado una función de Matlab que recibe como parámetros la función a integrar, una referencia al método y una lista de puntos en el eje X que servirán de intervalos.

\begin{lstlisting}[language=Matlab, caption={Función que aplica el método de integración en los intervalos.},captionpos=b,texcl=true]
% Para aplicar diversos métodos en el apartado B
function res = aplicarMetodo(func, met, X)
    res = 0;
    for i = 1:length(X)-1
        res = res + met(func, X(i), X(i+1));
    end
end
\end{lstlisting}

\subsection{Apartado A}
Aplicamos los métodos al cálculo de \(\int_{0}^{1} e^x \,dx\) y los comparamos con el valor exacto computado con:
\begin{lstlisting}
fA=@(x) exp(x);
exactoA = integral(fA,0,1); % valor exacto e^x en [0,1]
\end{lstlisting}

Podemos apreciar que el incremento en el número de intervalos aproxima el resultado a la solución exacta usando el mismo método y que, comparando la solución de cada método con el mismo número de intervalos (1000 en nuestro caso) con el valor exacto, vemos que el error cometido es mucho menor con la fórmula de Simpson que aproxima cuadráticamente.
\begin{table}
\begin{center}
\begin{tabular}{ |c|c| } 
 \hline 
 Metodo & abs(exacto-resultado) \\ 
 \hline \hline
 Rectángulo izquierda &  $8.59\times10^{-04}$ \\ 
 \hline
 Rectángulo derecha &  $8.59\times10^{-04}$ \\ 
 \hline
 Rectángulo punto medio &  $7.16\times10^{-08}$ \\ 
 \hline
 Trapecio &  $1.4319\times10^{-07}$ \\ 
 \hline
 Simpson &  $1.9984\times10^{-15}$ \\ 
 \hline
\end{tabular}
\end{center}
\caption{Error absoluto cometido para el apartado A.}
\end{table}


\subsection{Apartado B}
Aplicamos los métodos al cálculo de \(\int_{0}^{2\pi} cos(x^2 - 1) \,dx\) y los comparamos con el valor exacto y la aproximación computados con:
\begin{lstlisting}
fB=@(x) cos((x.^2) - 1);
X = 0:pi/1000:2*pi;
Y = arrayfun(fB,X);
aproxiB = trapz(X,Y);
exactoB = integral(fB, 0, 2*pi);
\end{lstlisting}

Usando esta vez 2000 intervalos obtenemos unas conclusiones similares al apartado A.
\begin{table}
\begin{center}
\begin{tabular}{ |c|c| } 
 \hline 
 Metodo & abs(exacto-resultado) \\ 
 \hline \hline
 Rectángulo izquierda &  $2.7603\times10^{-04}$ \\ 
 \hline
 Rectángulo derecha &  $2.615\times10^{-04}$ \\ 
 \hline
 Rectángulo punto medio &  $3.632\times10^{-06}$ \\ 
 \hline
 Trapecio &  $7.2638\times10^{-06}$ \\ 
 \hline
 Simpson &  $4.5356\times10^{-11}$ \\ 
 \hline
\end{tabular}
\end{center}
\caption{Error absoluto cometido para el apartado B.}
\end{table}